\documentclass[12pt]{article}

%packages
\usepackage[ngerman]{babel}
\usepackage[a4paper, total={7in, 10.2in}]{geometry}
\usepackage{enumitem}
\usepackage{tabularx}
\usepackage{makecell}
\usepackage{graphicx}

%general definitions
\pagestyle{empty}
\setlist{noitemsep}
{\renewcommand{\arraystretch}{1.5}%

\begin{document}
\subsection*{Persönliche Daten}
\begin{tabularx}{\textwidth}{l|l}
Vorname & Benedikt \\
Nachname & Schmidt \\
Adresse & Pradler Straße 69, Stiege 1, Innsbruck, Österreich \\
Familienstand & ledig \\
Geburtsdatum & 26. September 1990 \\
Staatsbürgerschaft & Österreich \\
Telefon & +4366488742636 \\
Email & benediktibk@gmail.com \\
Homepage & \url{http://me.benediktschmidt.at}
\end{tabularx}

\subsection*{Berufserfahrungen}
\begin{tabularx}{\textwidth}{l|l|l}
\thead{von} & \thead{bis} & \\
Oktober 2017 & heute & Leading Operations Engineer bei world-direct \\
Juni 2016 & November 2017 & \makecell[cl]{Selbstständiges Gewerbe für Dienstleistungen in der \\ automatischen Datenverarbeitung und Informationstechnik} \\
April 2015 & September 2017 & Development/Operations VoIP bei world-direct \\
Juli 2012 & März 2015 & \makecell[cl]{Studentische Hilfskraft am Sprachenzentrum der \\ Technischen Universität München} \\
September 2011 & September 2011 & Junior Software Developer bei Datacon Technology GmbH \\
Jänner 2011 & April 2011 & Junior Software Developer bei Datacon Technology GmbH \\
Juli 2008 & Juli 2008 & \makecell[cl]{Innovationspraktikant am Institut für Mathematik der \\ Universität Innsbruck}
\end{tabularx}

\subsection*{Ausbildung}
\begin{tabularx}{\textwidth}{l|l|l}
\thead{von} & \thead{bis} & \\
Februar 2016 & November 2016 & CCNP Collaboration \\
Oktober 2013 & März 2015 & \makecell[cl]{Master Elektro- und Informationstechnik an der \\ Technischen Universität München} \\
Mai 2011 & Oktober 2013 & \makecell[cl]{Bachelor Elektro- und Informationstechnik an der \\ Technischen Universität München} \\
September 2005 & Juni 2010 & HTL Elektrotechnik Anichstraße, Innsbruck
\end{tabularx}

\subsection*{Kompetenzen}
\begin{tabularx}{\textwidth}{l|l}
Programmiersprachen & C\#, C/C++, Python, VHDL, Javascript, Bash, Matlab, Powershell \\
Webdesign & HTML, CSS \\
CCNP Collaboration \\
Fundierte Netzwerkkenntnisse \\
Versionskontrolle & git, mercurial, Subversion \\
Anwendungen & Visual Studio, Wireshark \\
Betriebssysteme & Windows, Linux
\end{tabularx}

\end{document}
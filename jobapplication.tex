\documentclass[12pt]{article}

%packages
\usepackage{url}

%content
\begin{document}
Benedikt Schmidt
Pradler Straße 69, Stiege 1
Innsbruck
+4366488742636
benediktibk@gmail.com

Unternehmen ???
Adresse ???
Adresse ???

\section{Bewerbung als Softwareentwickler}
Ich bin derzeit auf der Suche nach neuen Herausforderung, und bin während der Recherche auf Ihr Unternehmen gestoßen. Die dort gebotene Vielfalt an Möglichkeiten, sowie auch insbesondere die technischen Herausforderungen, sprechen mich an. Dementsprechend möchte ich mich hiermit als Softwareentwickler bei Ihnen bewerben. \par
Meine bisherigen Erfahrungen sind sehr breit gefächert. Aktuell bin ich Teamleiter (Leading Operations Engineer) bei dem Unternehmen world-direct. Dort führe ich inhaltlich ein Team von 9 Softwareentwicklern und operativen Mitarbeitern, welches sich mit der Weiterentwicklung und dem Betrieb einer der größten Cisco basierten Voice over IP Lösung in Mitteleuropa beschäftigt. Der Fokus dieses Produkts ist es am KMU-Markt möglichst viele Features einer Voice over IP Lösung im Vollausbau anbieten zu können, und dabei aber gleichzeitig die hohen initialen Investitionskosten pro Kunde zu vermeiden. \par
Die Probleme die sich mir dabei im Berufsalltag stellen sind sehr vielfältig, auch insbesondere dadurch, dass Betrieb und Weiterentwicklung dem Team obliegen. Damit bin ich in der spannenden Situation die Entwicklung eines neuen Features von der Produktidee bis hin zum Support im laufenden Betrieb begleiten zu können. \par
Begonnen habe ich bei dem Unternehmen world-direct als DevOp in jenem Team, dass ich nun seit 2017 leite. Zuvor schloss ich das Masterstudium der Elektro- und Informationstechnik an der Technischen Unversität München mit Auszeichnung ab. Vor diesem Studium besuchte ich die Höhere Technische Lehranstalt Anichstraße in Innsbruck, Abteilung Elektrotechnik, welche ich ebenfalls mit ausgezeichnetem Erfolg abschloss. \par
Meine Ausbildung war somit eher auf der elektrotechnischen Seite, jedoch lag mein Fokus stets auf Softwareentwicklung. Viele der Probleme die sich in der Elektrotechnik stellen können mit entsprechenden Algorithmen und Software gelöst werden, weshalb ich während meiner Ausbildung stets Gelegenheit hatte Softwareentwicklung praktisch anzuwenden. Diese Intention war auch der Grund weshalb ich mich für ein Studium der Elektro- und Informationstechnik anstelle von Informatik entschied. Ich betrachte Softwareentwicklung als anspruchsvolles und komplexes Handwerk, welches es in der Praxis zu erlernen gilt. Die notwendigen theoretischen Grundlagen konnte ich mir immer entweder im Zuge des Studiums an der Technischen Universität München, oder auch im Selbsstudium, aneignen. \par
Aufgrund meiner Fähigkeiten und Interessen bin ich davon überzeugt, dass ich ein wertvolles Mitglied in Ihrem Team sein könnte und würde mich freuen, wenn wir uns bei einem persönlichen Gespräche kennen lernen könnten. \par
Falls Sie mehr zu meiner Person erfahren möchten bitte ich Sie darum einen Blick auf meinen Lebenslauf zu werfen. Detaillierte Informationen finden sich auch auf meiner Homepage \url{http://me.benediktschmidt.at}. \par
Mit freundlichen Grüßen,
Benedikt Schmidt
\end{document}